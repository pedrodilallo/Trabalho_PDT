\documentclass[11pt]{article}
\usepackage[utf8]{inputenc}
\usepackage[T1]{fontenc}
\usepackage[brazil]{babel}
\usepackage{newtxtext} % Times New Roman
\usepackage[top=3cm, bottom=2cm, left=3cm, right=2cm]{geometry}
\usepackage{ragged2e}
\usepackage{enumitem}
\usepackage{setspace}
\usepackage{makeidx}
\usepackage{graphicx}
\usepackage{afterpage}
\makeindex % Criar índice depois


% Início do corpo do texto
\begin{document}
\justifying % Texto justificado
\onehalfspacing % Espaçamento de 1,5 linha
\setlength{\parindent}{0cm}  % Indentação dos parágrafos
\renewcommand*\familydefault{\rmdefault}
%-----------------------------------------------------------%
% Capa                                
%-----------------------------------------------------------%
\thispagestyle{empty}
\begin{center}
\includegraphics[scale=0.6]{logo-dep.jpg}\\
\vspace*{.8cm}
{\huge \textbf{UNIVERSIDADE FEDERAL DE SÃO CARLOS}}\\
\vspace*{.8cm}
{\Large \textbf{PROJETO DE TRABALHO}}\\
\vspace*{3cm}
{\Large \textbf{ANÁLISE E PROJETO DE SITUAÇÃO REAL DE PROJETO}}\\
\vspace*{4.5cm}
\begin{flushright}
    \onehalfspacing
    {\Large  Paulo Roberto Mattielo Filho - 792323}\\
    {\Large  Pedro Peverari di Lallo - 792328}\\
    {\Large  Matheus Ossent de Oliveira - 792305 }\\
    \vspace*{.3cm}
    {\Large \textbf{Professor:}}
    {\Large Dr. João Alberto Camarotto}\\
\end{flushright}
\vspace*{\fill}
{\large \bf SÃO CARLOS / 2023}
\end{center}

\newpage{}

\tableofcontents


\newpage{}

\section{Identificação e reconhecimento da situação estudada}



\subsection{Caracterização da Empresa}



\subsection{Situação Estudada}



\subsection{Instrumentos de Trabalho}



\subsection{Layout}



\subsection{Sequência das atividades}



\subsection{Pessoal}



\subsection{Tarefas}



\subsection{Rendimento e produtividade}



\subsubsection{Produção}



\subsubsection{Produtividade}



\subsubsection{Qualidade}



\subsection{Duração do trabalho}



\subsection{Ambiente Físico-Químico do Trabalho}



\subsection{Saúde dos Operadores}




%%%%%%%%%%%%%%%%%%%%%%%%%%%%%%%%%%%%%%%%%%%%%%%%%%%%%%%%%
\section{Análise e Diagnóstico}



\end{document}