\documentclass[11pt]{article}
\usepackage[utf8]{inputenc}
\usepackage[T1]{fontenc}
\usepackage[brazil]{babel}
\usepackage{newtxtext} % Times New Roman
\usepackage[top=3cm, bottom=2cm, left=3cm, right=2cm]{geometry}
\usepackage{ragged2e}
\usepackage{enumitem}
\usepackage{setspace}
\usepackage{makeidx}
\usepackage{graphicx}
\usepackage{afterpage}
\makeindex % Criar índice depois


% Início do corpo do texto
\begin{document}
\justifying % Texto justificado
\onehalfspacing % Espaçamento de 1,5 linha
\setlength{\parindent}{0cm}  % Indentação dos parágrafos
\renewcommand*\familydefault{\rmdefault}
%-----------------------------------------------------------%
% Capa                                
%-----------------------------------------------------------%
\thispagestyle{empty}
\begin{center}
\includegraphics[scale=0.6]{logo-dep.jpg}\\
\vspace*{.8cm}
{\huge \textbf{UNIVERSIDADE FEDERAL DE SÃO CARLOS}}\\
\vspace*{.8cm}
{\Large \textbf{PROJETO DE TRABALHO}}\\
\vspace*{3cm}
{\Large \textbf{ANÁLISE E PROJETO DE SITUAÇÃO REAL DE PROJETO}}\\
\vspace*{4.5cm}
\begin{flushright}
    \onehalfspacing
    {\Large  Paulo Roberto Mattielo Filho - 792323}\\
    {\Large  Pedro Peverari di Lallo - 792328}\\
    {\Large  Matheus Ossent de Oliveira - 792305 }\\
    \vspace*{.3cm}
    {\Large \textbf{Professor:}}
    {\Large Dr. João Alberto Camarotto}\\
\end{flushright}
\vspace*{\fill}
{\large \bf SÃO CARLOS / 2023}
\end{center}

\newpage{}

\tableofcontents


\newpage{}

\section{Identificação e reconhecimento da situação estudada}



\subsection{Caracterização da Empresa}

Nome: A fábrica pizzaria.

Setor: Alimentício.


Fornecedores: Atacados e produtores laticínios - variados, sem contratos fixos, fornecem os ingredientes para montar as pizzas, como farinhas, molhos e queijos.

Clientes: Consumidores B2C.

Horário de funcionamento: 18 às 23 todos os dias. 

Horário de trabalho: Das 7 (preparação e fracionamento dos insumos) às 16 e  das 18 às 23 (preparação das massas e das pizzas em si) todos os dias.

Produto: Pizzas.

Exigência de qualidade: Pizzas de nível alto a um preço baixo, consumidores prezam pelo custo benefício e o estabelecimento apresenta rigorosos padrões de qualidade. O ponto principal é não apresentar variabilidade na quantidade de recheio, nem em sua qualidade. As massas por sua vez apresentam problemas quanto ao periodo de maturação. 


\subsection{Situação Estudada}

Produção da massa para as pizzas, bem como preparação, organização dos ingredientes e preparo da pizza em si.


\subsection{Instrumentos de Trabalho}

\begin{itemize}
    \item Forno; 
    \item Esteira;
    \item Luvas;
    \item Toucas;
    \item Conchas para pegar molho;
    \item Bancadas individuais
    \item Bandeijas com ingredientes;
    \item Cortador de pizza;
    \item Cortador de presuntos;
    \item abridora de massas;
    \item Câmara fria;
    \item Computadores para recebimento de pedidos;
    \item Furadores; 
    \item Espátulas;
  \end{itemize}

Forno/ Esteira/ Luvas/ Toucas/ Conchas/ Facas/ Bancadas individuais/Camara fria/ computadores para recebimento de pedidos/ Balcão onde são separadas pizzas para delivery.;


\subsection{Layout}

O layout da fábrica pizzaria inclui um forno em esteira que permite a produção contínua de pizzas. Há uma área de preparação de ingredientes, uma área de armazenamento de alimentos refrigerados e uma área de expedição. O forno em esteira é colocado na cozinha e é utilizado para cozinhar as pizzas. O layout é projetado para facilitar a produção e entrega de pizzas.


\subsection{Sequência das atividades}



\subsection{Pessoal}

10 pessoas envolvidas no preparo/ 5 envolvidas em atividades administrativas;


\subsection{Tarefas}

Preparação da massa realizada em uma máquina de mistura onde os ingredientes (açúcar, água, sal, farinha, fermento). Os ingredientes se mantêm sempre em uma bancada ao lado do misturador, dessa forma o operador consegue adicioná-los facilmente sem se movimentar pela fábrica (da pizza). Após pronta, a massa é embalada e armazenada num resfriador. Depois, cada um dos pizzaiolos abrem a massa maturada e então adicionam os ingredientes já preparados individualmente por um cozinheiro utilizado só para isso.  


\subsection{Rendimento e produtividade}



\subsubsection{Produção}

A pizzaria apresenta, segundo o Dono do Estabelecimento, uma produção de 200 a 250 pizzas por dia. Entretando, foi comentado que, em algumas ocasiões, chegam a ser vendidas mais de mil pizzas em um só dia.

\subsubsection{Produtividade}

\begin{table}[htbp]
\caption{}
\begin{tabular}{|l|l|r|}
\hline
Aferição do rendimento e produtividade: & Número De unidades vendidas/dia & 200 \\ \hline
 & Número de funcionários & 10 \\ \hline
 & Período de trabalho/dia & 7 \\ \hline
 & Número de fornos & 2 \\ \hline
 & Tempo de uso do forno & 4,5 \\ \hline
 & Tempo por pizza no forno & 1,5 \\ \hline
 & Produtividade por trabalhador & 20 \\ \hline
 & Produtividade Horas-Homem & 2,857142857 \\ \hline
 & Ocupação da máquina & 80,00\% \\ \hline
 & Produtividade do processo & 28,57142857 \\ \hline
\end{tabular}
\label{}
\end{table}

\newpage{}

\subsubsection{Qualidade}

A empresa não apresenta indicadores para o controle da qualidade dos produtos. 

\subsection{Duração do trabalho}

O período vai das 18 às 23 com ritmo variável de acordo com a demanda (que apresenta picos aos finais de semana).


\subsection{Ambiente Físico-Químico do Trabalho}

A pizzaria que fizemos a análise possui um ambiente de trabalho ativo e movimentado. Como esperado, a presença do forno e da cozinha resulta em temperaturas elevadas, exigindo dos trabalhadores uma boa resistência ao calor. Porém, a iluminação é boa, o que permite uma visibilidade clara na cozinha, o que é importante para garantir a qualidade e precisão nas atividades realizadas. Para não apresentar problemas agudos relativos ao calor, a pizzaria apresenta sistema de ventilação ativa, que apesar de útil apresenta gargalo, dada a grande potência do forno. 

Os “montadores” das pizzas não ficam fixos em suas bancadas, tendo que ir buscar o pedido, colado no lado oposto do salão, atrás das atendentes. Isso faz com que o tempo seja desperdiçado e com que o ambiente (em horários de pico) fique bem movimentado. A produção de pizzas é contínua e o ambiente pode ser agitado devido ao fluxo constante de clientes e à necessidade de atender aos pedidos rapidamente. Isto exige dos trabalhadores uma combinação de habilidades culinárias e físicas, além de uma disposição para trabalhar em um ambiente desafiador.


\subsection{Saúde dos Operadores}

Sem problemas de saúde decorrentes do trabalho.



%%%%%%%%%%%%%%%%%%%%%%%%%%%%%%%%%%%%%%%%%%%%%%%%%%%%%%%%%
\section{Análise e Diagnóstico}



\end{document}